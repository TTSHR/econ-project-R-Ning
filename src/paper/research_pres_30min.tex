\documentclass[11pt]{beamer}
% \documentclass[11pt,handout]{beamer}
\usepackage[T1]{fontenc}
\usepackage[utf8]{inputenc}
\usepackage{float, afterpage, rotating, graphicx}
\usepackage{epstopdf}
\usepackage{longtable, booktabs, tabularx}
\usepackage{fancyvrb, moreverb, relsize}
\usepackage{eurosym, calc, chngcntr}
\usepackage{amsmath, amssymb, amsfonts, amsthm, bm} 

\usepackage[backend=biber, natbib=true, bibencoding=inputenc, bibstyle=authoryear-ibid, citestyle=authoryear-comp, maxcitenames=3, maxbibnames=10]{biblatex}
\setlength{\bibitemsep}{1.5ex}
\addbibresource{refs.bib}

\hypersetup{colorlinks=true, linkcolor=black, anchorcolor=black, citecolor=black, filecolor=black, menucolor=black, runcolor=black, urlcolor=black}

\setbeamertemplate{footline}[frame number]
\setbeamertemplate{navigation symbols}{}
\setbeamertemplate{frametitle}{\centering\vspace{1ex}\insertframetitle\par}


\begin{document}

\title{GIFT-GIVING PROJECT}

\author[Ning Chen] % \& others]
{
{\bf Ning Chen}\\
{\small ESMT}\\[1ex]
%{\bf Add other authors}\\
%{\small and affilitations}\\[1ex]
}

\date{
{\bf Berlin, Germany}\\
{\small $25.08.2015$}
}


\begin{frame}
    \titlepage
    \note{~}
\end{frame}


\begin{frame}[t]
    \frametitle{Gift-giving behavior?}
    \begin{itemize}
        \item<+-> Please cite this template as: \citet{GaudeckerEconProjectTemplates}
        \item<+-> Gift-giving behavior is regarded as a tool for reinforcing relationships within society (\cite{Caplow1982}).
        \item<+-> A great deal of prior research has examined the psychological and environmental factors influencing the gift-giver's choice and purchase of a gift. For example...
        \item<+-> Many studies have investigated the underlying motivations of gift-giving behavior (\cite{Goodwin1990}; \cite{Wolfinbarger1990}; \cite{sherry1983}).
    \end{itemize}
    \note{~}
\end{frame}


\begin{frame}[t]
\frametitle{The effects of gift-giving motivations?}
\begin{itemize}
\item<+-> Influence the gift selection process (\cite{Goodwin1990}).
\item<+-> Influence the gift-giver's expectation of reciprocity (\cite{Goodwin1990}).
\item<+-> Influence the characteristics of gifts (\cite{Goodwin1990}).
\item<+-> The question of how the underlying motivations of gift-giving behavior influence the gift-giver's subsequent purchase is still open!
\end{itemize}
\note{~}
\end{frame}


\begin{frame}[t]
\frametitle{Our goal?}
\begin{itemize}
\item<+-> Recent research suggests that prior decisions for choices may influence consumers' subsequent preferences (\cite{Khan2006}).
\item<+-> Studies in social cognition have investigated the underlyng cognitive processes in understanding how prior tasks affect following behaviors (\cite{demarree2005}).
\item<+-> We intend to find out how the motivations of giving a gift affect gift-giver's self-indulgence in a subsequent purchase opportunity.
\end{itemize}
\note{~}
\end{frame}



\begin{frame}[t]
\frametitle{Motivations underlying gift-giving behavior?}
\begin{itemize}
\item<+-> Sherry $(1983)$ has proposed that altruistic motive and agonistic motive are the two underlying motivations of gift-giving behavior.
\item<+-> Additional research suggested the existence of an obligatory motive of gift-giving (Caplow $1982, 1984)$.
\item<+-> Goodwin, Smith and Spiggle $(1990)$ state that an obligatory motive and a voluntary motive are the two endpoints of gift-giving motivation.
\item<+-> Goodwin, Smith and Spiggle $(1990)$ state that an obligatory motive and a voluntary motive are the two endpoints of gift-giving motivation.
\end{itemize}
\note{~}
\end{frame}


\begin{frame}[t]
\frametitle{Two basic motivations}
\begin{itemize}
\item<+-> Altruistic motive: When the gift-giver tries to maximize the receiver’s pleasure (Sherry, $1983$).
\item<+-> Complying-with-social-norm motive: When gift-giving behavior is regarded as the norms, social responsibility, and reciprocity (Caplow $1982, 1984$).
\end{itemize}
\note{~}
\end{frame}


\begin{frame}[t]
\frametitle{Indulgence?}
\begin{itemize}
\item<+-> Indulgence is closely related to both luxury and hedonics, often involving spending on items perceived as luxuries relative to one’s means (\cite{kivetz2002}).
\item<+-> Indulgences are inferior to necessities in the hierarchy of needs (\cite{maslow1970}).
\item<+-> Consuming indulgences can cause "pain of paying" (\cite{prelec1998}) and can evoke feelings of guilt and regret (\cite{strahilevitz1998}).
\end{itemize}
\note{~}
\end{frame}


\begin{frame}[t]
\frametitle{Contextual factors influencing indulgence}
\begin{itemize}
\item<+-> Donation to charity, prior shopping restraints, invested efforts, presenting hedonic and utilitarian alternatives singely or perceiving oneself as excelling in the previous task.
\item<+-> Altruistic acts and intents have been demonstrated as incentives for self-indulgence.
\item<+-> Prosocial spending increases donor’s happiness (\cite{Dunn2008}). Happiness, in turn, has been shown to promote self-indulgence (\cite{Perry1985}; \cite{Rosenhan1974}; \cite{Mischel1968}).
\end{itemize}
\note{~}
\end{frame}



\begin{frame}[t]
\frametitle{Hypotheses $1$, $2$ and $3$}
\begin{itemize}
\item<+-> H$1$: Consumers will indulge more after giving a gift with an altruistic (vs. social norm) motivation.
\item<+-> H$2$: Happiness will mediate the relationship between gift-giving motivations and self-indulgence.
\item<+-> H$3$: Consumers with an altruistic gift-giving motive will indulge more than consumers with a social norm motive when they perceive self-indulgence as morally appropriate, but not when they perceive self-indulgence as morally inappropriate.
\end{itemize}
\note{~}
\end{frame}



\begin{frame}[t]
\frametitle{Study: Design}
\begin{itemize}
\item<+-> Participants: $N = 101$
\item<+-> Design
\begin{itemize}
\item<+-> Gift-giving motivation (altruism vs. social norm) $*$ Perception of morality
\item<+-> DV: Willingness to indulge
\end{itemize}
\end{itemize}
\note{~}
\end{frame}



\begin{frame}[t]
\frametitle{Study: Results}
\begin{figure}
\centering
\includegraphics[width=.6\textwidth]{../../out/figures/figure1}
\end{figure}
\end{frame}


\begin{frame}[t]
\frametitle{Conclusions}
\begin{itemize}
\item<+-> Consumers will indulge more after giving a gift with an altruistic (vs. social norm) motivation.
\item<+-> Happiness will mediate the relationship between gift-giving motivations and self-indulgence.
\item<+-> Consumers with an altruistic gift-giving motive will indulge more than consumers with a social norm motive when they perceive self-indulgence as morally appropriate, but not when they perceive self-indulgence as morally inappropriate.
\end{itemize}
\note{~}
\end{frame}



\begin{frame}[t]
\frametitle{Discussions}
\begin{itemize}
\item<+-> Our contributions
\begin{itemize}
\item<+-> We contribute to the literature on the effects of prior behaviors on consumers’ subsequent preferences.
\item<+-> We contribute to the gift-giving literature by manipulating gift-giving motivations in experiments and showing how they affect subsequent behavior.
\end{itemize}
\item<+-> Limitations
\item<+-> Future Research
\end{itemize}
\note{~}
\end{frame}


\begin{frame}[t]
\frametitle{THANK YOU!}
\note{~}
\end{frame}









% Print black screen only in presentation mode for finishing up.
\mode<beamer> {
    \beamersetaveragebackground{black}
    \begin{frame}
        \frametitle{}
    \end{frame}

    \beamersetaveragebackground{white}
}

\begin{frame}[allowframebreaks]
    \frametitle{References}
    \renewcommand{\bibfont}{\normalfont\footnotesize}
    \printbibliography
\end{frame}

\end{document}
