\documentclass[11pt, a4paper, leqno]{article}
\usepackage{a4wide}
\usepackage[T1]{fontenc}
\usepackage[utf8]{inputenc}
\usepackage{float, afterpage, rotating, graphicx}
\usepackage{epstopdf}
\usepackage{longtable, booktabs, tabularx}
\usepackage{fancyvrb, moreverb, relsize}
\usepackage{eurosym, calc, chngcntr}
\usepackage{amsmath, amssymb, amsfonts, amsthm, bm}
\usepackage{caption}
\usepackage{mdwlist}
\usepackage{xfrac}
\usepackage{setspace}
\usepackage{xcolor}
\usepackage{subcaption}
\usepackage{minibox}
\usepackage{indentfirst}
% \usepackage{pdf14} % Enable for Manuscriptcentral -- can't handle pdf 1.5
% \usepackage{endfloat} % Enable to move tables / figures to the end. Useful for some submissions.
\usepackage[
    natbib=true,
    bibencoding=inputenc,
    bibstyle=authoryear-icomp,
    citestyle=authoryear-icomp,
    maxcitenames=3,
    maxbibnames=10,
    useprefix=false,
    sortcites=true,
    backend=biber
]{biblatex}
\AtBeginDocument{\toggletrue{blx@useprefix}}
\AtBeginBibliography{\togglefalse{blx@useprefix}}
\setlength{\bibitemsep}{1.5ex}
\addbibresource{refs.bib}

\usepackage[unicode=true]{hyperref}
\hypersetup{
    colorlinks=true,
    linkcolor=black,
    anchorcolor=black,
    citecolor=black,
    filecolor=black,
    menucolor=black,
    runcolor=black,
    urlcolor=black
}


\widowpenalty=10000
\clubpenalty=10000

\setlength{\parskip}{1ex}
\setlength{\parindent}{2ex}
\setstretch{1.5}


\begin{document}

\title{GIFT-GIVING PROJECT
\thanks{Ning Chen: ESMT, Schlossplatz $1$, $10178$ Berlin, Germany.\href{mailto:ning.chen@esmt.org} {\nolinkurl{ning.chen [at] esmt [dot] org}}, tel.$+49 30212310$.}
% subtitle: 
% \\[1ex] 
% \large Subtitle here
}

\author{Ning Chen
% \\[1ex]
% Additional authors here
}

\date{
{\bf Preliminary -- please do not quote} 
\\[1ex] 
\today
}

\maketitle


\begin{abstract}
This research suggests that different gift-giving motivations influence a gift-giver’s self-indulgence differently. Gift-givers with an altruistic motivation tend to be more self-indulgent than gift-givers with a compliance-with-social-norm motive. We find that this effect is mediated by happiness and moderated by the perceived appropriateness of self-indulgence.
\end{abstract}
\clearpage

\section{Introduction} % (fold) 
\label{sec:introduction}
If you are using this template, please cite this item from the references: (\cite{GaudeckerEconProjectTemplates}). \par
\setlength{\parindent}{2ex} 
Imagine you go to a shopping mall to buy a birthday gift and, after buying the gift, you spend some time shopping for yourself. Would the fact that you buy a gift for someone else influence what you want to purchase for yourself? This research investigates whether different gift-giving motivations will influence a gift-giver’s self-indulgence differently. \par
Gift-giving behavior is regarded as a tool for reinforcing relationships within society (\cite{Caplow1982}). \citeauthor{Wolfinbarger1990} (\citeyear{Wolfinbarger1990}) classifies gift-giving motivations into three different categories: an altruistic motive, a self-interested motive and a compliance-with-social-norm motive. Gift-giving motivations have been shown to affect the gift selection process, the gift-giver’s expectations of reciprocity, as well as the characteristics of gifts (\cite{Goodwin1990}). However, the question of how the underlying motivations of gift-giving behavior influence the gift-giver’s subsequent purchase is still open. \par
Recent research suggests that altruistic acts and intents are incentives for self-indulgence (\cite{Khan2006}). Meanwhile, \citeauthor{Dunn2008} (\citeyear{Dunn2008}) show that prosocial spending increases the donor’s happiness. Happiness, in turn, has been shown to promote self-indulgence (\cite{Mischel1968, Perry1985, Rosenhan1974}). However, happiness facilitates (inhibits) self-indulgence when indulgences are perceived as morally appropriate (inappropriate; (\cite{Perry1985})). \par
In the current research, we build on these findings and examine how the motivations of giving a gift affect gift-giver’s self-indulgence in the following purchase. Results of our study show that giving a gift with an altruistic motivation increases a gift-giver’s self-indulgence relative to giving a gift with a compliance-with-social-norm motivation. The results also suggest that happiness mediates this effect. Additionally, we demonstrate that consumers with an altruistic gift-giving motive will indulge more than consumers with a social norm motive when they perceive self-indulgence as morally appropriate, but not when they perceive self-indulgence as morally inappropriate. \par

%%%%%%%%%%%%%%%%%%%%%%%%%%%%
\section{Study}
\label{sec:study}

\setlength{\parindent}{2ex}
Our study mainly investigated whether gift-givers with an altruistic motivation are more (less) willing to indulge in the purchase of luxury brands relative to gift-givers with a compliance-with-social-norm motive when indulgence is perceived as morally appropriate (inappropriate). In addition, we tested happiness as the underlying mechanism. \par

\subsection{Method}

\setlength{\parindent}{2ex}
\emph{Design and Procedure.} One hundred and one people were recruited from a university lab to participate in this study in exchange for a 5-euro show-up fee. The independent variables in this study are consumers' gift-giving motivation (altruism vs. compliance-with-social-norm) and their perception of whether indulgence is moral or immoral. Participants were randomly assigned to one of the two gift-giving motivation conditions. And the perception of morality was measured as a continuous variable. \par
In this computer-based experiment, participants were asked to finish a survey about their shopping behaviors carefully. In the survey, there were four ostensibly unrelated tasks. In task 1, each participant read a hypothetical scenario of giving a gift according to one of the two manipulation conditions and tried to experience the scenario as vividly as possible. For example, the scenario of altruism condition read, "Recall that as compensation for our participation in this experiment, you will get 5 euros. However, you realize that it's your coworker's birthday tomorrow. Although you are not so close to him, you still wish he can have a nice birthday. To make him happy, you will buy him a birthday present." After reading these instructions, participants answered a question which was used to make the manipulation stronger. Then, participants finished two manipulation check scales. In task 2, participants evaluated on the Emotion Scale ($\alpha= .93$) which consists of four emotion items (happy, thrilled, cheerful, excited). We measured the proposed mediator, happiness, by averaging the scores of these four emotion items. In the third task, participants rated their willingness to buy 21 luxury brands ($\alpha= .91$) such as Lacoste, Boss ($1=$ I would not be willing to purchase, $7=$ I would be willing to purchase). The results of this scale served as the dependent measure of this study (self-indulgence). The more they were willing to buy the luxury brands, the more indulgent they were. In the fourth task, participants were asked to rate their perceptions of whether self-indulgence is morally appropriate or inappropriate on a 7-point Likert scale ($1=$ I think they are morally inappropriate, $7=$ I think they are morally appropriate). \par

\subsection{Results}

\setlength{\parindent}{2ex}
\emph{Manipulation Checks.} We eliminated two participants whose responses in willingness to buy luxury brands showed extreme values (boxplot: beyond inner fences). As predicted, the gift-giving motivation manipulations had significant impacts on people's motivations. Participants in the altruism condition ($M= 5.19$) perceived themselves as being more altruistic than participants in the social norm condition ($M= 4.76$; $F(1, 97)= 5.01$, $p= .027$). Participants in the social norm condition ($M= 4.76$) perceived themselves as more obeying social norms than participants in the altruism condition ($M= 3.75$; $F(1, 97)= 9.88$, $p< .01$). \par

\emph{Self-indulgence.} Results showed that there was a significant interaction between the perception of morality and gift-giving motivation ($B= .29$, $t(95)= 1.97$, $p= .051$). To further understand this interaction, we computed the slopes of the two fitting lines of the altruism condition and the social norm condition. The more indulgence was perceived as moral, the more participants in the altruism condition tended to indulge ($B= .28$; $t(48)= 2.63$, $p= .011$; see fig.$1$). But for participants in the social norm condition, their willingness to buy luxury brands was not influenced by the perception of morality ($B= -.016$; $t(47)= -.16$, NS). \par
Spotlight analysis did not show significant differences at $+/-1$ SD, but the means are in the predicted direction: When indulgence was perceived as moral ($+1$ SD), participants in the altruism condition ($M= 4.00$) indulged more than participants in the social norm condition ($M= 3.59$). But when indulgence was perceived as immoral ($-1$ SD), participants in the altruism condition ($M= 3.08$) indulged less than participants in the social norm condition ($M= 3.65$). \par

\emph{Mediating Effect.} Given the significant influence of gift-giving motivations on gift-giver's self-indulgence, we examined whether happiness was the mediator in this relationship. With the method of bootstrapping mediation, we included the gift-giving motivations as independent variables, self-indulgence as dependent variable, and happiness as mediator. The results illustrated that the effect of gift-giving motivations on gift-givier's self-indulgence was mediated by happiness ($0$ was not included in the $95\%$ confidence interval; $[.0553, .5369]$). It means that there is a significant indirect effect of gift-giving motivations on self-indulgence through happiness. \par



\captionsetup[figure]{labelfont=bf,justification=centering}
\captionsetup[figure]{name=FIGURE}
\begin{figure}
\centering
\caption{\\STUDY$1$: EFFECT OF THE PERCEPTION OF MORALITY ON SELF-INDULGENCE}
\includegraphics[width=.8\textwidth]{../../out/figures/figure1}
%\caption*{\label{fig:FIGURE 1}  }
\end{figure}




\section{General Discussion}
\setlength{\parindent}{2ex}
We found that different gift-giving motivations have different influences on gift-givers’ self-indulgence through happiness. We contribute to the literature on the effects of prior behaviors on consumers’ subsequent preferences by being the first to show that different motivations underlying gift-giving behavior influence consumers’ self-indulgence differently. This research also contributes to the gift-giving literature by manipulating gift-giving motivations in experiments. \par








 
\clearpage
\setstretch{1}
\printbibliography
\setstretch{1.5}

%\appendix
%\counterwithin{table}{section}
%\counterwithin{figure}{section}

\end{document}
